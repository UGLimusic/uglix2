% uglix v0.2
%
% 2020-07
%
% PARTIE SPECIFIQUE A LA CLASSE BOOK

%---------Pages

\fancypagestyle{plain}{%
\fancyhf{} % clear all header and footer fields
\fancyfoot[C]{\begin{tikzpicture}
  	%\node (shadow) {};
  	\node[rounded corners=2pt,fill=pagenumber@color,inner ysep=5pt, inner xsep=2ex,text=white] at (0,0) %at ($(shadow)+(-2pt,2pt)$) 
  	{\titlefont\thepage};
  	\end{tikzpicture}} % except the center
\renewcommand{\headrulewidth}{0pt}
\renewcommand{\footrulewidth}{0pt}}
\setlength{\footskip}{40pt}
\pagestyle{plain}


%------------creative commons header
\newcommand\creativecommonsheader{\vspace*{\fill}
\begin{minipage}{\textwidth}\vspace*{3em}
\rule{\textwidth}{0.4pt}\vspace*{1em}
\double{\textit{Ce document est mis à disposition selon les termes de la licence Creative Commons « Attribution - Pas d’utilisation commerciale - 
Partage 
dans les 
mêmes conditions 3.0 non transposé ».}}{\includegraphics[width=6cm]{C:/texlive/2020/texmf-dist/tex/latex/uglix/images/cc.png}}{6cm}
%/usr/local/texlive/2019/texmf-dist/tex/latex/UGLiLuaLaTeX/cc.png
\end{minipage}
\fancyfoot[R]{}
}
%------------creative commons footer

\newcommand\creativecommonsfooter{
\fancypagestyle{plain}{
	\fancyfoot[L]{\ \\[-1.3em]\tiny\color{lightgray}Lycée Rabelais - Saint Brieuc}
\fancyfoot[R]{\includegraphics[width=2cm]{C:/texlive/2020/texmf-dist/tex/latex/uglix/images/cc.png}}
%/usr/local/texlive/2020/texmf-dist/tex/latex/UGLiLuaLaTeX/cc.png
\fancyfoot[C]{\begin{tikzpicture}
  	\node[rounded corners=3pt,fill=shadow@color,inner ysep=5pt, inner xsep=2ex,text=shadow@color] (shadow) {\titlefont\thepage};
  	\node[rounded corners=3pt,fill=pagenumber@color,inner ysep=5pt, inner xsep=2ex,text=white] at ($(shadow)+(-2pt,2pt)$) 
  	{\titlefont\thepage};
  	\end{tikzpicture}} % except the center
\renewcommand{\headrulewidth}{0pt}
\renewcommand{\footrulewidth}{0pt}}
\setlength{\footskip}{40pt}
\pagestyle{plain}}

%-------------LONGUEURS

\newlength\decalnumsec\setlength{\decalnumsec}{-0.6em}

\newlength\decalnumsubsec\setlength{\decalnumsubsec}{-0.4em}
\newlength\decalxtitlesubsec\setlength{\decalxtitlesubsec}{.53em}

\newlength\decalnumsubsubsec\setlength{\decalnumsubsubsec}{-0.33em}
\newlength\decalxtitlesubsubsec\setlength{\decalxtitlesubsubsec}{.67em}

%-------------FORMATAGE NUMERATION

\renewcommand{\thesection}{\Roman{section}}
\renewcommand{\thesubsection}{\arabic{subsection}}
\renewcommand{\thesubsubsection}{\alph{subsubsection}}
\setcounter{secnumdepth}{3}

%-------------FORMATAGE SECTIONS

\def\@intro{}
\newcommand{\intro}[1]{\def\@intro{\og#1\fg}}

%Intro
%
\newcommand{\introduction}[1]{\vspace*{-4em}\begin{flushright} \tikz{\draw[bleufonce@color!50] (0,0) 
node{\textit{\footnotesize«#1»}};}\end{flushright}}
%

%
%\exostart
%
\newcommand{\exostart}
	{
	\newpage\pagecolor{subsection@color!5}
	\tikz{\draw (0,0) node{\huge\titlefont\color{subsection@color} Exercices};}
	}

%
%\correcstart
%
\newcommand{\correcstart}
	{
	\newpage\pagecolor{vertfonce@color!2}
	\tikz{\draw (2pt,-2pt) node{\huge\titlefont\color{shadow@color} Corrigés des exercices};
	\draw (0,0) node{\huge\titlefont\color{vertfonce@color} Corrigés des exercices};}
	}

%--Chapter
\titleformat{\chapter}[block]{\vspace*{-3cm}\fontsize{35pt}{42pt}\selectfont\titlefont\color{chapter@color}}
{\LARGE Chapitre \thechapter \\[-1em]}
{0pt}{}

%--Sections
\titleformat{\section}[block]
{\LARGE\titlefont\color{section@color}}
{\hspace*{-.5em}
	\raisebox{\decalnumsec}
	{
		\begin{tikzpicture}
			\node (numsec) {\LARGE\titlefont\thesection};
			\fill[rounded corners=2pt,fill=section@color] (numsec.north west) -- ($(numsec.north east)$) -- (numsec.south east) -- 
			($(numsec.south 
			west)$) -- cycle;
			\node[white] at (numsec) {\LARGE\titlefont\thesection};
		\end{tikzpicture}
	}
}
{0pt}
{}

%--Subsections
\titleformat{\subsection}[block]
{\hspace*{\decalxtitlesubsec}\large\titlefont\color{subsection@color}}
{\hspace*{-.7em}
	\raisebox{\decalnumsubsec}
	{%
	\begin{tikzpicture}
	\node (numsec) {\large\titlefont\thesubsection};
	\fill[rounded corners=2pt,fill=subsection@color] (numsec.north west) -- ($(numsec.north east)$) -- (numsec.south east) -- 
	($(numsec.south west)$) -- cycle;
	\node[white] at (numsec) {\large\titlefont\thesubsection};
	\end{tikzpicture}
	}
}
{0pt}
{}

%--Subsubsections
\titleformat{\subsubsection}[block]
{\hspace*{\decalxtitlesubsubsec}\titlefont\color{subsubsection@color}}
{\hspace*{-.9em}
\raisebox{\decalnumsubsubsec}
{%
\begin{tikzpicture}
\node (numsec) {\normalsize\titlefont\thesubsubsection};
\fill[rounded corners=2pt,fill=subsubsection@color] (numsec.north west) -- ($(numsec.north east)$) -- (numsec.south east) -- 
($(numsec.south west)$) -- cycle;
\node[white] at (numsec) {\normalsize\titlefont\thesubsubsection};
\end{tikzpicture}
}
}
{0pt}
{}