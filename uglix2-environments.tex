% uglix v0.2
%
% 2020-07
%
%
% COMMANDES D'ENVIRONNEMENTS

\renewenvironment{leftbar}[1][\hsize]
{
  \def\FrameCommand
    {%
        {\color{chapter@color}\vrule width 1pt}%
        \hspace{0pt}%must no space.
        \fboxsep=\FrameSep\colorbox{white}%
    }%
    \MakeFramed{\hsize#1\advance\hsize-\width\FrameRestore}%
}
{\endMakeFramed}
\newenvironment{demonstration}
     {\textbf{Preuve:}\begin{leftbar}}
     {\hfill$\blacksquare$\end{leftbar}}
%
\geometry{scale=.75}
% Macro to draw the shape behind the text, when it fits completly in the
% page
\newcommand*\parchmentframe[1]{%
	\tikz{%
		\node[inner sep=1ex] (A) {#1};  % Draw the text of the node
		\begin{scope}[on background layer]  % Draw the shape behind
			\fill[normal border] (A.south east) -- (A.south west) -- (A.north west) -- (A.north east) -- cycle;
		\end{scope}
	}%
}
% Macro to draw the shape, when the text will continue in next page
\newcommand*\parchmentframetop[1]{%
	\tikz{%
		\node[inner sep=1ex] (A) {#1};    % Draw the text of the node
		\begin{scope}[on background layer]
			\fill[normal border]              % Draw the ``complete shape'' behind
			(A.south east) -- (A.south west) -- (A.north west) -- (A.north east) -- cycle;
			\fill[torn border]                % Add the torn lower border
			($(A.south east)-(0,.2)$) -- ($(A.south west)-(0,.2)$) -- ($(A.south west)+(0,.2)$) -- ($(A.south east)+(0,.2)$) -- cycle;
		\end{scope}
	}%
}
% Macro to draw the shape, when the text continues from previous page
\newcommand*\parchmentframebottom[1]{%
	\tikz{%
		\node[inner sep=1ex] (A) {#1};   % Draw the text of the node
		\begin{scope}[on background layer]
			\fill[normal border]             % Draw the ``complete shape'' behind
			(A.south east) -- (A.south west) -- (A.north west) -- (A.north east) -- cycle;
			\fill[torn border]               % Add the torn upper border
			($(A.north east)-(0,.2)$) -- ($(A.north west)-(0,.2)$) -- ($(A.north west)+(0,.2)$) -- ($(A.north east)+(0,.2)$) -- cycle;
		\end{scope}
	}%
}
% Macro to draw the shape, when both the text continues from previous page
% and it will continue in next page
\newcommand*\parchmentframemiddle[1]{%
	\tikz{%
		\node[inner sep=1ex] (A) {#1};   % Draw the text of the node
		\begin{scope}[on background layer]
			\fill[normal border]             % Draw the ``complete shape'' behind
			(A.south east) -- (A.south west) -- (A.north west) -- (A.north east) -- cycle;
			\fill[torn border]               % Add the torn lower border
			($(A.south east)-(0,.2)$) -- ($(A.south west)-(0,.2)$) -- ($(A.south west)+(0,.2)$) -- ($(A.south east)+(0,.2)$) -- cycle;
			\fill[torn border]               % Add the torn upper border
			($(A.north east)-(0,.2)$) -- ($(A.north west)-(0,.2)$) -- ($(A.north west)+(0,.2)$) -- ($(A.north east)+(0,.2)$) -- cycle;
		\end{scope}
	}%
}
%________________________________
%
% --- Environnement Modulable

\newenvironment{modulbox}[2]% #1 : color #2 : title
{
	\tikzset{
		normal border/.style={#1!15},
		torn border/.style={#1!15,decorate, decoration={random steps,segment length=.2cm,amplitude=1pt}},
	}
	\def\FrameCommand{\parchmentframe}%
	\def\FirstFrameCommand{\parchmentframetop}%
	\def\LastFrameCommand{\parchmentframebottom}%
	\def\MidFrameCommand{\parchmentframemiddle}%
	\vskip\baselineskip
	\MakeFramed {\advance\hsize-\width\FrameRestore}%
	\noindent\tikz{\node[inner sep=1ex, rounded corners=2pt,draw=#1,fill=#1, anchor=west, overlay] at ( 0ex, 1ex) 
	{\color{white} \titlefont #2};}\par\vspace*{.5em}
}
{
%\vspace*{.5em}
\endMakeFramed}

% --- Environnement Modulable avec icone


\newenvironment{modulboxg}[3]% #1 : color #2 : title
{
	\tikzset{
		normal border/.style={#1!15},
		torn border/.style={#1!15,decorate, decoration={random steps,segment length=.2cm,amplitude=1pt}},
	}
	\def\FrameCommand{\parchmentframe}%
	\def\FirstFrameCommand{\parchmentframetop}%
	\def\LastFrameCommand{\parchmentframebottom}%
	\def\MidFrameCommand{\parchmentframemiddle}%
	\vskip\baselineskip
	\MakeFramed {\advance\hsize-\width\FrameRestore}%
	\noindent\tikz
	{
		\node[inner sep=1ex, rounded corners=2pt,draw=#1,fill=#1, anchor=west, overlay] at ( 0ex, .8ex) 
			{\color{#1}\hspace*{3.8ex}\titlefont #2};
		\node[overlay] at ( 2ex, .8ex) {\includegraphics[height=1.8em]{#3}};
		\node[overlay,anchor=west] at ( 3.8ex, .6ex) {\color{white}\titlefont #2};
	}\par\vspace*{.5em}
}
{
	%\vspace*{.5em}
	\endMakeFramed}



% --- Environnements DIVERS COURS

\newenvironment{aretenir}{\begin{modulbox}{UGLiGreen}{À retenir}}{\end{modulbox}}
\newenvironment{attention}{\begin{modulbox}{UGLiRed}{Attention}}{\end{modulbox}}
\newenvironment{notation}[1][]{\begin{modulbox}{def@color}{Notation#1}}{\end{modulbox}}
\newenvironment{definition}[1][]{\begin{modulbox}{def@color}{Définition#1}}{\end{modulbox}}
\newenvironment{propriete}[1][]{\begin{modulbox}{prop@color}{Propriété#1}}{\end{modulbox}}
\newenvironment{methode}[1][]{\begin{modulbox}{meth@color}{Méthode#1}}{\end{modulbox}}
\newenvironment{remarque}[1][]{\begin{modulbox}{rem@color}{Remarque#1}}{\end{modulbox}}
\newenvironment{theoreme}[1][]{\begin{modulbox}{thm@color}{Théorème#1}}{\end{modulbox}}
\newenvironment{exemple}[1][]{\begin{modulbox}{exe@color}{Exemple#1}}{\end{modulbox}}
\newenvironment{consequence}[1][]{\begin{modulbox}{subsection@color}{Conséquence#1}}{\end{modulbox}}


% --- Environnements DIVERS


\newenvironment{encadre}[1][]{\begin{modulbox}{gray}{#1}}{\end{modulbox}}
\newenvironment{encadrecolore}[2]{\begin{modulbox}{#2}{#1}}{\end{modulbox}}

% --- Exos et correc

\newcounter{exonumber}
\setcounter{exonumber}{0}
\newenvironment{exercice}[1][]{\stepcounter{exonumber}\begin{modulbox}{exo@color}{Exercice \theexonumber #1
}}{\end{modulbox}}
\newcounter{correcnumber}
\setcounter{correcnumber}{0}

\newenvironment{exercicecorrection}[1][]{\stepcounter{correcnumber}\begin{modulbox}{vertfonce@color}{Corrigé de l'exercice \thecorrecnumber #1
	}}{\end{modulbox}}


% gardé pour compatibilité
\newenvironment{correction}[1][]{\stepcounter{correcnumber}\begin{modulbox}{vertfonce@color}{Corrigé de l'exercice \thecorrecnumber #1
}}{\end{modulbox}}

%
%
% --- Environnement  Enoncé
%

\excludecomment{enonce}
\newcommand{\cacheEnonce}{\excludecomment{enonce}}
\newcommand{\afficheEnonce}{\includecomment{enonce}}


% --- Corrige

\excludecomment{corrige}
\newcommand{\cacheCorrige}{\excludecomment{corrige}}
\newcommand{\afficheCorrige}{\includecomment{corrige}}