\documentclass[a4paper,12pt,french]{book}
\usepackage[margin=2cm]{geometry}
\usepackage[thinfonts,latinmath]{uglix2}

\begin{document}
\chapter{Titre Chapitre}
\chapter*{Titre Chapitre sans numéro}

\section{Titre section}
\section*{Titre section sans numéro}

\subsection{Titre sous-section}
\subsection*{Titre sous-section sans numéro}
\subsubsection{Titre sous-sous-section}
\subsubsection*{Titre sous-sous-section sans numéro}

\newpage

\section*{Environnements de base}

\begin{aretenir}
	bla bla
\end{aretenir}

\begin{attention}
	contenu...
\end{attention}

\begin{notation}[s]
	Le [s] pour le pluriel
\end{notation}

\begin{definition}[ : droite]
	le [ : droite] pour continuer le titre
\end{definition}

\begin{propriete}
	contenu
\end{propriete}

\begin{methode}
	contenu.
\end{methode}

\begin{remarque}
	contenu.
\end{remarque}

\begin{theoreme}
	contenu.
\end{theoreme}

\begin{exemple}
	contenu.
\end{exemple}

\begin{consequence}
	contenu
\end{consequence}


\newpage

\section*{Environnements numérotés automatiquement}

\begin{exercice}[ : inégalité de Machin-Truc]
	bla bla.
\end{exercice}

\begin{exercice}[ : Restes chinois]
	Bla.
\end{exercice}

\begin{exercicecorrection}
	bla bla
\end{exercicecorrection}

\begin{exercicecorrection}
	bla bla
\end{exercicecorrection}

\newpage

\section*{Environnement masquables}


\afficheEnonce

\begin{enonce}
	Énoncé Affiché
\end{enonce}

\cacheEnonce

\begin{enonce}
	Énoncé Caché
\end{enonce}

\afficheEnonce


\afficheCorrige
\begin{corrige}
\begin{encadre}[Réponse]
	Corrigé affiché
\end{encadre}
\end{corrige}

\cacheCorrige
\begin{corrige}
\begin{encadre}[Réponse]
		Corrigé caché
\end{encadre}
\end{corrige}

\newpage

Voici du \textsc{Python} en ligne : \pythoninline{print("tu as ",age," ans")}. Joli non ?\\

De même \algoinline{a ← a + 1} est du pseudocode en ligne.

\begin{pythoncode}
print('Hello')
\end{pythoncode}

De l'invite de commande \textsc{Python} :

\begin{pythonshell}
>>> a = 2 * 3
6
\end{pythonshell}

\begin{algo}
a ← 3
\end{algo}

\end{document}